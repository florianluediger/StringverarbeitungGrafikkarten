\documentclass[a4paper,11pt,twoside,ngerman,color]{book}
\usepackage[a4paper,left=3.5cm,right=2.5cm,bottom=3.5cm,top=3cm]{geometry}

\usepackage[german,english]{babel}

\usepackage[pdftex]{graphicx,color}
\usepackage{amsmath,amssymb,subfigure}


% Theorem-Umgebungen
\usepackage[amsmath,thmmarks]{ntheorem}

% Korrekte Darstellung der Umlaute
\usepackage[utf8]{inputenc}
\usepackage[T1]{fontenc}

% Codeabschnitte
\usepackage{listings}

\definecolor{mygreen}{rgb}{0,0.6,0}
\definecolor{mygray}{rgb}{0.5,0.5,0.5}
\definecolor{mymauve}{rgb}{0.58,0,0.82}

\lstset{ %
	backgroundcolor=\color{white},   % choose the background color
	basicstyle=\ttfamily\footnotesize,        	 % size of fonts used for the code
	breaklines=true,                 % automatic line breaking only at whitespace
	captionpos=b,                    % sets the caption-position to bottom
	commentstyle=\color{mygreen},    % comment style
	escapeinside={\%*}{*)},          % if you want to add LaTeX within your code
	keywordstyle=\color{blue},       % keyword style
	stringstyle=\color{mymauve},     % string literal style
	frame=single,
	showstringspaces=false,
	tabsize=2,
	numbers=left
}

\lstdefinelanguage{MyC++} {
	language=C++,
	numbers=left,
	stepnumber=1,
	numbersep=10pt,
}

% Bibtex deutsch
\usepackage[backend=biber]{biblatex}
\addbibresource{bibliography.bib}

% URLs
\usepackage{url}

% Caption Packet
\usepackage[margin=0pt,font=small,labelfont=bf]{caption}

% Zeichnen von Darstellungen
\usepackage{tikz}
\usepackage{pgfplots}
\usepackage{pgfplotstable}

\usetikzlibrary{positioning}

% Hinzufügen weiterer Symbole
\usepackage{bbding}

% Zeilenabstand einstellen %
\renewcommand{\baselinestretch}{1.25}

% Floating-Umgebungen anpassen %
\renewcommand{\topfraction}{0.9}
\renewcommand{\bottomfraction}{0.8}

% Abkürzungen richtig formatieren %
\usepackage{xspace}
\newcommand{\vgl}{vgl.\@\xspace} 
\newcommand{\zB}{z.\nolinebreak[4]\hspace{0.125em}\nolinebreak[4]B.\@\xspace}
\newcommand{\bzw}{bzw.\@\xspace}
\newcommand{\dahe}{d.\nolinebreak[4]\hspace{0.125em}h.\nolinebreak[4]\@\xspace}
\newcommand{\etc}{etc.\@\xspace}
\newcommand{\evtl}{evtl.\@\xspace}
\newcommand{\ggf}{ggf.\@\xspace}
\newcommand{\bzgl}{bzgl.\@\xspace}
\newcommand{\so}{s.\nolinebreak[4]\hspace{0.125em}\nolinebreak[4]o.\@\xspace}
\newcommand{\iA}{i.\nolinebreak[4]\hspace{0.125em}\nolinebreak[4]A.\@\xspace}
\newcommand{\sa}{s.\nolinebreak[4]\hspace{0.125em}\nolinebreak[4]a.\@\xspace}
\newcommand{\su}{s.\nolinebreak[4]\hspace{0.125em}\nolinebreak[4]u.\@\xspace}
\newcommand{\ua}{u.\nolinebreak[4]\hspace{0.125em}\nolinebreak[4]a.\@\xspace}
\newcommand{\og}{o.\nolinebreak[4]\hspace{0.125em}\nolinebreak[4]g.\@\xspace}
\newcommand{\oBdA}{o.\nolinebreak[4]\hspace{0.125em}\nolinebreak[4]B.\nolinebreak[4]\hspace{0.125em}d.\nolinebreak[4]\hspace{0.125em}A.\@\xspace}
\newcommand{\OBdA}{O.\nolinebreak[4]\hspace{0.125em}\nolinebreak[4]B.\nolinebreak[4]\hspace{0.125em}d.\nolinebreak[4]\hspace{0.125em}A.\@\xspace}

% Leere Seite ohne Seitennummer, naechste Seite rechts
\newcommand{\blankpage}{
 \clearpage{\pagestyle{empty}\cleardoublepage}
}

% Keine einzelnen Zeilen beim Anfang eines Abschnitts (Schusterjungen)
\clubpenalty = 10000
% Keine einzelnen Zeilen am Ende eines Abschnitts (Hurenkinder)
\widowpenalty = 10000 \displaywidowpenalty = 10000
% EOF
