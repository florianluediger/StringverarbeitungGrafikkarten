\chapter{Einfacher, paralleler String-Vergleich}

Um einige Techniken zur String-Verarbeitung in kompilierten Anfragepipelines auf Grafikkarten entwickeln zu können, wird zunächst ein Operator für den einfachen String-Vergleich für den Query Compiler erarbeitet.
Ein Vergleich auf Gleichheit stellt dabei die einfachste, sinnvolle Variante von String-Verarbeitung dar, wodurch der Einfluss vieler Eigenschaften von Strings auf die Ausführung entsprechender Operationen leicht untersucht werden kann.

Zunächst wird dazu die Motivation hinter einer derartigen Untersuchung erläutert und eine bestehende Technik zur String-Verarbeitung erklärt.
Außerdem wird eine Umsetzung des einfachen String-Vergleichs mittels der CUDA-Schnittstelle ohne spezielle Optimierungen vorgestellt und für einen alternativen Workload für weitere Tests leicht angepasst werden.
Schließlich wird beurteilt, ob die Lösung das Potential hat eine optimale Performanz zu bieten und auf einen Nachteil der einfachen Implementierung eingegangen.

\section{Motivation}

In den meisten modernen Anwendungsfällen für Datenbankmanagementsysteme stellt die Verarbeitung von String-Daten einen wesentlichen Bestandteil dar.
So wird häufig auf Gleichheit, Ungleichheit, das Enthaltensein eines Teilstrings oder das Erfüllen eines regulären Ausdrucks getestet.
Die effiziente Berechnung unterschiedlicher Operatoren auf Zeichenketten ist somit essentiell für das Erreichen eines hohen Durchsatzes und für das Gewährleisten von maximaler Performanz.
In diesem Kontext versprechen Grafikkarten durch ihre hochgradig parallele Architektur in der Theorie eine bestmögliche Leitung.

Bisherige Ansätze zur Verarbeitung von Zeichenketten auf Grafikkarten greifen auf das Konzept der Dictionaries zurück.
Dabei wird eine Tabelle mit allen Strings aufgebaut und zu diesen ein Schlüssel abgespeichert, welcher zusammen mit jedem String in den anderen Tabellen der Datenbank gespeichert wird.
Somit können String-Operationen durch andere Operationen auf den Schlüsseln abgebildet werden, wodurch diese eine einheitliche Struktur und damit ein effizientes Ausführungsmuster auf Grafikkarten erhalten.
Das Aufbauen und Verwalten des für diese Technik verwendeten Dictionaries erzeugt vor allem bei Daten, die sich häufig ändern, einen hohen Aufwand, worunter die Leistungsfähigkeit des Gesamtsystems sinkt.

Um diesen Verwaltungsaufwand für eine zusätzliche Datenstruktur zu eliminieren, wird eine Lösung entwickelt, die direkt auf den Zeichenketten arbeitet und trotzdem eine hohe Performanz bietet.

\section{Umsetzung des einfachen String-Vergleichs}
\label{sec:equals_umsetzung}

Als Basis für die Untersuchungen wird der String-Vergleich-Operator für den Query Compiler zunächst naiv, also ohne tiefgehende Optimierungen umgesetzt.
Mithilfe dieses Operators kann in einer Datenbankabfrage beispielsweise eine Selektion über eine Spalte mit Spring-Daten durchgeführt werden.
Die Anforderung des Operators besteht somit darin, eine Liste von Zeichenketten mit einem vorher spezifizierten String zu vergleichen und zu entscheiden, ob diese identisch sind oder nicht.

Zur Durchführung dieser Operation wird jedem Thread der GPU eine Zeichenkette aus dem Datensatz zugewiesen.
Zunächst wird überprüft, ob die Länge des Strings mit der des Suchstrings übereinstimmt, sodass der entsprechende Eintrag direkt verworfen werden kann.
Sind die Längen identisch, werden beide Zeichenketten Zeichen für Zeichen durchlaufen und diese an jeder Stelle auf Gleichheit überprüft.
Sobald eine Ungleichheit gefunden wurde, wird ein entsprechendes Flag gesetzt und die weiteren Zeichen müssen nicht mehr genauer betrachtet werden.

Sämtliche Threads innerhalb eines Warp werden entsprechend dem Verarbeitungsmodell der GPU parallel abgearbeitet.
Somit sind die Positionen, an denen die Strings verglichen werden ebenfalls für alle Threads identisch.
Sobald der Vergleichsstring im gesamten Warp vollständig durchlaufen wurde, wird das Zwischenergebnis geschrieben.
Sollten alle Threads in dem Warp vorzeitig feststellen, dass keiner der Strings mit dem Suchstring übereinstimmt, wird die aktuelle Untersuchung vorzeitig abgebrochen.

Schließlich wird jedem Thread eine neue Zeichenkette aus dem Datensatz zugewiesen, sodass das Verfahren im weiteren Verlauf wiederholt wird.
Sobald der gesamte Datensatz durchlaufen wurde, ist die Berechnung abgeschlossen.

\tikzset{t_label/.style={rectangle,draw,minimum size=0.4cm,minimum width=2.1cm,inner sep=0pt}}
\tikzset{t_length/.style={rectangle,draw,minimum size=0.4cm,inner sep=0pt,fill=blue!30}}
\tikzset{t_active/.style={rectangle,draw,minimum size=0.4cm,inner sep=0pt}}
\tikzset{t_inactive/.style={rectangle,draw,minimum size=0.4cm,inner sep=0pt,fill=black!30}}
\tikzset{t_blank/.style={rectangle,minimum size=0.4cm,inner sep=0pt}}

\begin{figure}[ht]
	\centering
	\begin{tikzpicture}
	\begin{scope}
	\node (l1) {\textsf{Datensatz:}};
	\node[below = 0cm of l1] (l2) {\textsf{OMA, OPA, OTTO,}};
	\node[below = 0cm of l2] (l3) {\textsf{AHA, OMA, OHM,}};
	\node[below = 0cm of l3] (l4) {\textsf{OHA, DBMS, SQL}};
	\end{scope}
	\begin{scope}[xshift=4cm]
	\node[t_label] (0s) {\textsf{Suchstr.}};
	\node[t_label,below = 0.1cm of 0s] (0a) {\textsf{Thread 1}};
	\node[t_label,below = 0.1cm of 0a] (0b) {\textsf{Thread 2}};
	\node[t_label,below = 0.1cm of 0b] (0c) {\textsf{Thread 3}};
	
	\node[t_length, right = 0.3cm of 0s] (1s) {\textsf{3}};
	\node[t_length, right = 0.3cm of 0a] (1a) {\textsf{3}};
	\node[t_length, right = 0.3cm of 0b] (1b) {\textsf{3}};
	\node[t_length, right = 0.3cm of 0c] (1c) {\textsf{4}};
	
	\node[t_active, right = 0cm of 1s] (2s) {\textsf{O}};
	\node[t_active, right = 0cm of 1a] (2a) {\textsf{O}};
	\node[t_active, right = 0cm of 1b] (2b) {\textsf{O}};
	\node[t_inactive, right = 0cm of 1c] (2c) {};
	
	\node[t_active, right = 0cm of 2s] (3s) {\textsf{M}};
	\node[t_active, right = 0cm of 2a] (3a) {\textsf{M}};
	\node[t_active, right = 0cm of 2b] (3b) {\textsf{P}};
	\node[t_inactive, right = 0cm of 2c] (3c) {};
	
	\node[t_active, right = 0cm of 3s] (4s) {\textsf{A}};
	\node[t_active, right = 0cm of 3a] (4a) {\textsf{A}};
	\node[t_inactive, right = 0cm of 3b] (4b) {};
	\node[t_inactive, right = 0cm of 3c] (4c) {};
	
	\node[t_blank, right = 0cm of 4s] (5s) {};
	\node[t_active, right = 0cm of 4a,fill=green!30] (5a) {\Checkmark};
	\node[t_active, right = 0cm of 4b,fill=red!30] (5b) {\XSolid};
	\node[t_active, right = 0cm of 4c,fill=red!30] (5c) {\XSolid};
	
	\node[t_length, right = 0.1cm of 5s] (6s) {\textsf{3}};
	\node[t_length, right = 0.1cm of 5a] (6a) {\textsf{3}};
	\node[t_length, right = 0.1cm of 5b] (6b) {\textsf{3}};
	\node[t_length, right = 0.1cm of 5c] (6c) {\textsf{3}};
	
	\node[t_active, right = 0cm of 6s] (7s) {\textsf{O}};
	\node[t_active, right = 0cm of 6a] (7a) {\textsf{A}};
	\node[t_active, right = 0cm of 6b] (7b) {\textsf{O}};
	\node[t_active, right = 0cm of 6c] (7c) {\textsf{O}};
	
	\node[t_active, right = 0cm of 7s] (8s) {\textsf{M}};
	\node[t_inactive, right = 0cm of 7a] (8a) {};
	\node[t_active, right = 0cm of 7b] (8b) {\textsf{M}};
	\node[t_active, right = 0cm of 7c] (8c) {\textsf{H}};
	
	\node[t_active, right = 0cm of 8s] (9s) {\textsf{A}};
	\node[t_inactive, right = 0cm of 8a] (9a) {};
	\node[t_active, right = 0cm of 8b] (9b) {\textsf{A}};
	\node[t_inactive, right = 0cm of 8c] (9c) {};
	
	\node[t_blank, right = 0cm of 9s] (10s) {};
	\node[t_active, right = 0cm of 9a,fill=red!30] (10a) {\XSolid};
	\node[t_active, right = 0cm of 9b,fill=green!30] (10b) {\Checkmark};
	\node[t_active, right = 0cm of 9c,fill=red!30] (10c) {\XSolid};
	
	\node[t_length, right = 0.1cm of 10s] (11s) {\textsf{3}};
	\node[t_length, right = 0.1cm of 10a] (11a) {\textsf{3}};
	\node[t_length, right = 0.1cm of 10b] (11b) {\textsf{4}};
	\node[t_length, right = 0.1cm of 10c] (11c) {\textsf{3}};
	
	\node[t_active, right = 0cm of 11s] (12s) {\textsf{O}};
	\node[t_active, right = 0cm of 11a] (12a) {\textsf{O}};
	\node[t_inactive, right = 0cm of 11b] (12b) {};
	\node[t_active, right = 0cm of 11c] (12c) {\textsf{S}};
	
	\node[t_active, right = 0cm of 12s] (13s) {\textsf{M}};
	\node[t_active, right = 0cm of 12a] (13a) {\textsf{H}};
	\node[t_inactive, right = 0cm of 12b] (13b) {};
	\node[t_inactive, right = 0cm of 12c] (13c) {};
	
	\node[t_blank, right = 0cm of 13s] (14s) {};
	\node[t_active, right = 0cm of 13a,fill=red!30] (14a) {\XSolid};
	\node[t_active, right = 0cm of 13b,fill=red!30] (14b) {\XSolid};
	\node[t_active, right = 0cm of 13c,fill=red!30] (14c) {\XSolid};
	
	\node[above = 0.1cm of 1s] (arr1) {};
	\node[above = 0.1cm of 14s] (arr2) {};
	\draw[->, thick] (arr1) -- (arr2) node[midway, above] {{\large\textsf{Ausführungszeit}}};
	\end{scope}
	
	\begin{scope}
	\node[below = 0.5cm of 1c] (laenge) {\textsf{Längenvergleich}};
	\node[below = -0.2cm of 1c] (laenge_arrow) {};
	\draw [->] (laenge) -- (laenge_arrow);
	
	\node[below right = -0.2cm and -0.08cm of 5c] (neue_daten_arrow) {};
	\node[below = 1.3cm of neue_daten_arrow, align = center] (neue_daten) {\textsf{Ergebnis schreiben,} \\ \textsf{neue Daten laden}};
	\draw [->] (neue_daten) -- (neue_daten_arrow);
	
	\node[below = 0.5cm of 14c] (abbruch) {\textsf{frühzeitiger Abbruch}};
	\node[below = -0.2cm of 14c] (abbruch_arrow) {};
	\draw [->] (abbruch) -- (abbruch_arrow);
	\end{scope}
	\end{tikzpicture}
	\caption{Funktionsweise des Algorithmus innerhalb eines Warps mit drei Threads}
	\label{equals_naiv_algorithmus}
\end{figure}

In Abbildung \ref{equals_naiv_algorithmus} ist der Ablauf des Algorithmus innerhalb eines Warps mit drei Threads dargestellt.
Es ist erkennbar, an welchen Stellen die Lanes inaktiv werden, wann die Berechnung frühzeitig abgebrochen werden kann und an welchen Stellen das Ergebnis geschrieben wird und neue Daten aus dem Datensatz geholt werden.

\newpage

\begin{lstlisting}[language=C++,
	caption=Naive Implementierung einer Selektion von Strings,
	label=naive_equals]
data_length = char_offset[loop_var+1] - char_offset[loop_var] - 1;

// if string lengths are unequal, discard
if (active && data_length != search_length)
	active = false;

int search_id = 0;

// iterate over strings completely or until they don't match anymore
while(active && search_id < search_length) {
	int data_id = search_id + char_offset[loop_var];
	
	// when strings don't match, inactivate the lane
	if (active && data_content[data_id] != search_string[search_id])
		active = false;
	
	search_id++;
}
\end{lstlisting}

In Listing \ref{naive_equals} ist die Implementierung des Operators dargestellt, welcher in einer kompilierten Anfragepipeline die Selektion eines String-Attributs durchführen kann.
Dieser würde beispielsweise die rot markierte Selektion in Listing \ref{pipelining_example_code} ersetzen, falls das \emph{prio}-Attribut in dem Beispiel aus Abbildung \ref{pipeline_example_plan} Strings enthalten würde und alle Bestellungen mit der Priorität \emph{HIGH} gesucht werden würden.
Nach Abschluss der Berechnung ist der Wert der \texttt{active}-Variable genau dann \texttt{true}, wenn der String mit dem gesuchten String übereinstimmt, sodass im Anschluss mit weiteren Operationen fortgefahren werden kann.

Diese Implementierung erwartet, dass einige Daten vorher vom Hauptspeicher in den Speicherbereich der GPU kopiert wurden und dort zur Verfügung stehen.
Der Datensatz, der mit dem Vergleichsstring abgeglichen werden soll, besteht aus einer Aneinanderreihung der entsprechenden Zeichenketten ohne Trennzeichen und ist in der Variable \texttt{data\_content} gespeichert.
Damit daraus die ursprünglichen Strings extrahiert werden können, gibt es das Feld \texttt{char\_offset}, welches Informationen über die Indizes der Einzelstrings innerhalb des Datensatzes enthält.
Ebenfalls muss natürlich ein Zeiger auf den Suchstring und dessen Länge in den entsprechenden Parametern \texttt{search\_string} und \texttt{search\_string\_length} vorhanden sein.
Um die Berechnung rechtzeitig vor Speicherüberschreitungen abbrechen zu können, wird schließlich noch die Variable \texttt{line\_count} benötigt, welche die Anzahl der Zeichenketten im Datensatz beschreibt.

Im ersten Schritt wird für einen String überprüft, ob dessen Länge mit der des Suchstrings übereinstimmt und dieser anderenfalls verworfen.
Sind die Längen identisch, wird in der Schleife über beide Zeichenketten iteriert, bis das Ende beider erreicht wurde, oder festgestellt wird, dass ein Zeichen aus dem Vergleichsstring nicht mit dem aus dem Suchstring übereinstimmt.
An dieser Stelle fällt die parallele Struktur des Kernels besonders auf, da die Zeichen aller Strings parallel von den Threads durchlaufen werden und diese erst aufhören, wenn der letzte Thread den ihm zugewiesenen Datensatz vollständig durchlaufen hat.
Ist die Schleife abgeschlossen oder vorzeitig abgebrochen worden, kann am Zustand der \texttt{active}-Variable abgelesen werden, ob die Strings übereinstimmen oder nicht.

\section{Präfixtest als alternativer Workload}

Als zusätzliche String-Operation für die ein simpler Algorithmus existiert, wurde neben dem exakten String-Vergleich auch ein Präfixtest implementiert.
Dabei soll geprüft werden, ob die Strings aus einer Datenbanktabelle einen vordefinierten Präfix besitzen.

Der Algorithmus arbeitet ähnlich wie das in Kapitel \ref{sec:equals_umsetzung} vorgestellte Verfahren zum exakten String-Vergleich, mit einem einzigen Unterschied.
Für jede Zeichenkette aus der Tabelle wird zunächst geprüft, ob diese länger als der gesuchte Präfix ist, anstatt zu prüfen, ob die Längen identisch sind.
Danach verfährt der Algorithmus wie gehabt, sodass wieder beide Strings Zeichen für Zeichen verglichen werden, bis das Ende des Such-Präfixes erreicht ist.
An der Implementierung in Listing \ref{naive_equals} ändert sich dementsprechend fast nichts, es muss nur die Bedingung in Zeile 4 durch \texttt{active \&\& data\_length >= search\_length} ersetzt werden.

Dadurch, dass die Prüfung auf exakte Längengleichheit entfällt, müssen je nach Anwendungsfall viele Zeichenketten weiter durchlaufen werden, da diese nicht schon im ersten Schritt ausgeschlossen werden können, wie bei dem exakten String-Vergleich.
Durch diesen Umstand und durch die Tatsache, dass ein Präfixtest in realen Systemen ein häufig gefragter Anwendungsfall ist, ergibt diese Implementierung einen interessanten alternativen Workload, den es zu untersuchen gilt.

\section{Schlechte Auslastung der GPU}

Das in diesem Kapitel vorgestellte, naive Verfahren zum einfachen String-Vergleich hat leider den Nachteil, dass es die Ressourcen der GPU nicht besonders effizient ausnutzt.
In Abbildung \ref{equals_naiv_algorithmus} sind in grau die inaktiven Threads dargestellt.
Diese sind inaktiv geworden, da erkannt wurde, dass sie nicht mit der gesuchten Zeichenkette übereinstimmen, da sie entweder eine unpassende Länge besitzen oder im Laufe des Vergleichs der einzelnen Zeichen ein Unterschied festgestellt wurde.
Die untersuchten Strings aus der Tabelle können völlig unterschiedlich geartet sein, weshalb es häufig vorkommt, dass einige Threads ihre Untersuchung bereits abgeschlossen oder vorzeitig unterbrochen haben, während andere Threads innerhalb des Warps noch lange weiter rechnen müssen.
Aufgrund des Programmiermodells von Grafikkarten, laufen die inaktiven Threads weiter synchron zu den aktiven Threads des Warps, dabei wird allerdings das Ergebnis verworfen und diese verrichten keine nutzbare Arbeit mehr.

Im schlimmsten Fall werden 31 Threads aus einem Warp Zeichenketten mit unpassender Länge zugewiesen und einem einzigen Thread eine mit dem Vergleichsstring übereinstimmende Zeichenkette zugeteilt.
Somit stellen 31 Threads im ersten Schritt fest, dass die Länge des Strings nicht mit der des Vergleichsstrings übereinstimmt und werden somit inaktiv.
Der Thread mit dem passenden String muss allerdings noch über jedes Zeichen iterieren, bevor sich der gesamte Warp neue Strings holen kann.
Es kann also sein, dass für die Dauer der Iteration über den Suchstring die GPU nur zu $1/32$ ausgelastet ist, weshalb die Performanz dieser Lösung zu wünschen übrig lässt.

Es wäre dagegen schön, wenn den inaktiven Threads dynamisch neue Arbeit zugewiesen werden könnte, sobald diese inaktiv geworden sind.
Da in dem Algorithmus lediglich zwei Zeichen an zwei Positionen verglichen werden und diese Position für jeden Thread unterschiedlich sein kann, könnte ein Thread, sobald er inaktiv geworden ist, sich eine neue Zeichenkette holen und mit dieser weiter arbeiten.
Mit diesem Vorgehen könnte zwar eine hohe Auslastung erreicht werden, da die Threads niemals wirklich inaktiv werden können, allerdings funktioniert dies nicht mit dem in Kapitel \ref{sec:pipelining} vorgestellten Pipelining-Modell.
Dies liegt daran, dass gegebenenfalls noch beliebig viele andere Operationen in der Pipeline vor dem String-Vergleich durchgeführt werden müssen, bevor eine neue Zeichenkette an den Thread übergeben wird.

In Kapitel \ref{sec:equals_lane_refill} wird eine weitere Technik vorgestellt, mit der die Auslastung der GPU verbessert werden kann.
Diese wird auch im Umfeld einer Anfragepipeline funktionieren und somit für den praktischen Einsatz geeignet sein.