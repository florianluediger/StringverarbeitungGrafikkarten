\chapter{Evaluation des parallelen Musterabgleichs}

Nachdem im letzten Kapitel gezeigt wurde, dass das Lane Refill-Verfahren eine erhöhte Leistungsfähigkeit des einfachen String-Vergleichs erreicht, bleibt nun zu untersuchen, wie sich die vorgestellten Methoden zum parallelen Musterabgleich damit verhalten.
Zunächst soll geklärt werden, welcher der vorgestellten Algorithmen ohne tiefgreifende Verbesserungen die beste Laufzeit erzielen kann.
Im Anschluss wird untersucht, welcher Algorithmus am meisten vom Lane Refill profitiert und ob sich die Beobachtungen für unterschiedlich geartete reguläre Ausdrücke ändert.
Die Testumgebung ist dabei identisch zu dem im Kapitel \ref{sec:equals_testumgebung} beschriebenen Aufbau.

\section{Verwendete Workloads und deren Merkmale}

Für die ersten Tests wurde der in \ref{sec:equals_evaluation_workloads} vorgestellte DBLP-Datensatz verwendet.
Dieser eignet sich optimal zur Untersuchung der Laufzeitverbesserung durch das Lane Refill, da dieser analytische Workload mit variabler Selektivität Analysen zu verschiedenen Stufen der Unterauslastung ermöglicht.
Außerdem bietet sich dieser Datensatz an, da ein Vergleich zwischen dem einfachen Präfixtest und dem Präfixtest mittels regulärer Ausdrücke durchgeführt werden kann.

Ein weiterer Workload wurde dem TPC-H-Benchmark entnommen, in dem eine Selektion über die \emph{Name}-Spalte der \emph{Products}-Relation durchgeführt wird.
Die enthaltenen Strings haben eine Länge zwischen 26 und 43 Zeichen, welche aus den Kleinbuchstaben \emph{a-z} und dem Leerzeichen bestehen.
Der Datensatz wurde einige male repliziert, sodass dieser 46.000.000 Tupel enthält.

\section{Vorstellung der Messergebnisse}

Abbildungen \ref{fig:regex_dblpANY_no_buffer} bis \ref{fig:regex_p_name} zeigen die Laufzeiten der verschiedenen Algorithmen in den entwickelten Benchmarks.
Für ein einfacheres Verständnis werden besondere Auffälligkeiten dazu aufgezeigt und im nächsten Abschnitt analysiert.

\subsection{Vergleich der Basisalgorithmen mit dem DBLP-Datensatz}
\label{sec:regex_evaluation_beobachtung_1}

Bei der ersten Messung soll überprüft werden, wie sich die in Kapitel \ref{sec:regex_naiv} vorgestellten Verfahren zur Durchführung eines einfachen Musterabgleichs ohne tiefgreifende Verbesserungen wie dem Lane Refill-Verfahren verhalten.
Untersucht werden hier die in Kapitel \ref{sec:regex_duchlaufen} beschriebene \emph{Flat}-Variante des von Ragel generierten Algorithmus, die in Kapitel \ref{sec:regex_alternativen} vorgestellte \emph{Table}-Variante dieses Verfahrens, die dort ebenfalls angesprochene \emph{LIKE}-Operation, welche aus DogQC entnommen wurde und die in Kapitel \ref{sec:prefixtest} vorgestellte Implementierung zum einfachen Präfixtest.
Der untersuchte reguläre Ausdruck passt auf eine bestimmte Zeichenfolge am Anfang jedes Strings und lässt keine Zeichen vor diesem Suchstring zu, danach sind dagegen beliebige Strings zulässig. 

\begin{figure}[ht]
	\centering
	\begin{tikzpicture}
		\begin{axis}[
			ybar,		% Säulendiagramm
			xlabel={Anteil Matches},	% Achsenbeschriftung
			xtick=data,
			xticklabels from table={daten/regex_dblpANY_no_buffer.csv}{Selectivity},	% Beschriftung der Markierungen 
			width=\textwidth,	% Diagrammbreite
			height=300, % Diagrammhöhe
			bar width=0.2cm,	% Breite der Säulen
			ylabel={Laufzeit (ms)},	% Achsenbeschritung
			ymin=0,		% Säulen stehen auf der X-Achse
			%legend style={at={(0,0)}}
			legend pos=north west,		% Legende am oberen linken Rand positionieren
			legend style={legend cell align=left} % Text linksbündig anordnen
			]
			\addplot table [x expr=\coordindex, y=flat]{daten/regex_dblpANY_no_buffer.csv}; 
			\addlegendentry{Flat};
			\addplot table [x expr=\coordindex, y=table]{daten/regex_dblpANY_no_buffer.csv};
			\addlegendentry{Table}
			\addplot table [x expr=\coordindex, y=equals]{daten/regex_dblpANY_no_buffer.csv};
			\addlegendentry{Präfix}
			\addplot table [x expr=\coordindex, y=like]{daten/regex_dblpANY_no_buffer.csv};
			\addlegendentry{Like}
		\end{axis}
	\end{tikzpicture}
	\caption{Laufzeit für Präfixtest mit Basisalgorithmen für den DBLP-Datensatz}
	\label{fig:regex_dblpANY_no_buffer}
\end{figure}

Bei den in Abbildung \ref{fig:regex_dblpANY_no_buffer} dargestellten Ergebnissen fällt zunächst auf, dass die Ausführungszeit aller Algorithmen für einen größeren Anteil Matches monoton ansteigt.
Die beiden Implementierungen, welche reguläre Ausdrücke verwenden, werden ab einem Anteil von 8\% kaum noch langsamer, wogegen die Geschwindigkeit der anderen beiden Verfahren dabei immer schneller sinkt.
Die \emph{Flat}-Variante ist zwischen 15\% und 22\% schneller als die \emph{Table}-Variante der Ragel-Umsetzung, wobei der Unterschied für mittelgroße Selektivitäten am größten ist.
Der \emph{LIKE}-Operator aus DogQC ist zwischen 1\% und 13\% schneller als der Präfixtest, wobei der Unterschied für einen höheren Anteil Matches größer wird.
Der Präfixtest ist bei einer Selektivität von unter 32\% zwischen 50\% und 65\% schneller als die \emph{Flat}-Variante.
Bei einem Anteil Matches von 64\% beträgt dieser Unterschied 37\%.

\subsection{Verbesserung der Algorithmen durch das Lane Refill}
\label{sec:regex_evaluation_beobachtung_2}

Die nachfolgende Untersuchung dient dazu, den Einfluss des Lane Refill auf die Varianten \emph{Flat} und \emph{Table} des regulären Musterabgleichs sowie den einfachen Präfixtest zu analysieren.
Der hier verwendete reguläre Ausdruck ist derselbe wie im vorherigen Abschnitt.

\begin{figure}[ht]
	\centering
	\begin{tikzpicture}
		\begin{axis}[
			ybar,		% Säulendiagramm
			xlabel={Anteil Matches},	% Achsenbeschriftung
			xtick=data,
			xticklabels from table={daten/regex_dblpANY_buffer.csv}{Selectivity},	% Beschriftung der Markierungen 
			width=\textwidth,	% Diagrammbreite
			height=300, % Diagrammhöhe
			bar width=0.2cm,	% Breite der Säulen
			ylabel={Verbesserung durch Lane Refill (\%)},	% Achsenbeschritung
			ymin=0,		% Säulen stehen auf der X-Achse
			%legend style={at={(0,0)}}
			legend pos=north west,		% Legende am oberen linken Rand positionieren
			legend style={legend cell align=left} % Text linksbündig anordnen
			]
			\addplot table [x expr=\coordindex, y=flat]{daten/regex_dblpANY_buffer.csv}; 
			\addlegendentry{Flat};
			\addplot table [x expr=\coordindex, y=table]{daten/regex_dblpANY_buffer.csv};
			\addlegendentry{Table}
			\addplot table [x expr=\coordindex, y=equals]{daten/regex_dblpANY_buffer.csv};
			\addlegendentry{Präfix}
		\end{axis}
	\end{tikzpicture}
	\caption{Verbesserungen der Algorithmen durch das Lane Refill für den DBLP-Datensatz}
	\label{fig:regex_dblpANY_buffer}
\end{figure}

Abbildung \ref{fig:regex_dblpANY_buffer} zeigt dazu die Verbesserung der Laufzeit, welche durch das Einbinden des Lane Refill-Verfahrens in die Basisalgorithmen erreicht wird.
Bei den beiden Verfahren, die auf dem regulären Musterabgleich basieren, ist eine bedeutend größere Verbesserung zu sehen, als bei dem Präfixtest.
Beide auf Ragel basierenden Algorithmen werden in ähnlichem Maße durch das Lane Refill verbessert, wobei die \emph{Flat}-Variante etwas mehr davon profitiert.
Bis zu einer Selektivität von 16\% beträgt die Verbesserung der beiden Verfahren für den Musterabgleich mehr als 50\% gegenüber der ursprünglichen Laufzeit.
Die Verbesserung aller untersuchter Algorithmen ist für einen mittelgroßen Anteil Matches am größten und fällt besonders für größere Anteile stark ab.

\subsection{Vergleich der optimierten Algorithmen}
\label{sec:regex_evaluation_beobachtung_3}

Schließlich werden die absoluten Laufzeiten der durch das Lane Refill verbesserten Verfahren untereinander und mit dem \emph{LIKE}-Operator verglichen.
Der reguläre Ausdruck bleibt derselbe wie in den letzten beiden Abschnitten beschrieben.

\begin{figure}[ht]
	\centering
	\begin{tikzpicture}
		\begin{axis}[
			ybar,		% Säulendiagramm
			xlabel={Anteil Matches},	% Achsenbeschriftung
			xtick=data,
			xticklabels from table={daten/regex_dblpANY_optimal.csv}{Selectivity},	% Beschriftung der Markierungen 
			width=\textwidth,	% Diagrammbreite
			height=300, % Diagrammhöhe
			bar width=0.2cm,	% Breite der Säulen
			ylabel={Laufzeit (ms)},	% Achsenbeschritung
			ymin=0,		% Säulen stehen auf der X-Achse
			%legend style={at={(0,0)}}
			legend pos=north west,		% Legende am oberen linken Rand positionieren
			legend style={legend cell align=left} % Text linksbündig anordnen
			]
			\addplot table [x expr=\coordindex, y=flat-buffer]{daten/regex_dblpANY_optimal.csv}; 
			\addlegendentry{Flat mit Lane Refill};
			\addplot table [x expr=\coordindex, y=table-buffer]{daten/regex_dblpANY_optimal.csv};
			\addlegendentry{Table mit Lane Refill}
			\addplot table [x expr=\coordindex, y=equals-buffer]{daten/regex_dblpANY_optimal.csv};
			\addlegendentry{Präfix mit Lane Refill}
			\addplot table [x expr=\coordindex, y=like]{daten/regex_dblpANY_optimal.csv};
			\addlegendentry{Like}
		\end{axis}
	\end{tikzpicture}
	\caption{Laufzeiten der optimierten Algorithmen für den DBLP-Datensatz}
	\label{fig:regex_dblpANY_optimal}
\end{figure}

Die in Abbildung \ref{fig:regex_dblpANY_optimal} dargestellten Laufzeiten aller Algorithmen steigen für einen höheren Anteil von Matches monoton an.
Im Allgemeinen liefert der durch das Lane Refill verbesserte Präfixtest die beste Laufzeit, was nur nicht für eine Selektivität von 64\% gilt.
Für eine steigende Anzahl passender Matches steigen die gepufferten Operationen besonders stark an, weshalb die ungepufferte \emph{LIKE}-Implementierung für hohe Selektivitäten eine vergleichsweise geringe Laufzeit bietet, obwohl diese für niedrige Selektivitäten zu den langsamsten gehört.
Die \emph{Flat}-Variante des regulären Musterabgleichs bietet stets eine geringere Laufzeit als die entsprechende \emph{Table}-Variante.

\subsection{Einfluss beliebiger Anfangszeichen in dem DBLP-Datensatz}
\label{sec:regex_evaluation_beobachtung_4}

Als nächstes wird untersucht, wie sich die Laufzeiten der Algorithmen entwickeln, wenn beliebige Zeichen vor dem gesuchten String stehen dürfen.
Der Datensatz bleibt dazu identisch zu den vorherigen Tests, es wird lediglich der reguläre Ausdruck dementsprechend erweitert, dass dieser beliebige Zeichen vor dem Suchstring zulässt.
Diese Modifikation führt dazu, dass der Präfixtest an dieser Stelle nicht mehr untersucht werden kann, da dieser keine beliebigen Zeichen unterstützt.

\begin{figure}[ht]
	\centering
	\begin{tikzpicture}
		\begin{axis}[
			ybar,		% Säulendiagramm
			xlabel={Anteil Matches},	% Achsenbeschriftung
			xtick=data,
			xticklabels from table={daten/regex_ANYdblpANY.csv}{Selectivity},	% Beschriftung der Markierungen 
			width=\textwidth,	% Diagrammbreite
			height=300, % Diagrammhöhe
			bar width=0.15cm,	% Breite der Säulen
			ylabel={Laufzeit (ms)},	% Achsenbeschritung
			ymax=800,
			ymin=0,		% Säulen stehen auf der X-Achse
			%legend style={at={(0,0)}}
			legend pos=north east,		% Legende am oberen linken Rand positionieren
			legend style={legend cell align=left} % Text linksbündig anordnen
			]
			\addplot table [x expr=\coordindex, y=flat]{daten/regex_ANYdblpANY.csv}; 
			\addlegendentry{Flat};
			\addplot table [x expr=\coordindex, y=flat-buffer]{daten/regex_ANYdblpANY.csv};
			\addlegendentry{Flat mit Lane Refill}
			\addplot table [x expr=\coordindex, y=table]{daten/regex_ANYdblpANY.csv};
			\addlegendentry{Table}
			\addplot table [x expr=\coordindex, y=table-buffer]{daten/regex_ANYdblpANY.csv};
			\addlegendentry{Table mit Lane Refill}
			\addplot table [x expr=\coordindex, y=like]{daten/regex_ANYdblpANY.csv};
			\addlegendentry{Like}
		\end{axis}
	\end{tikzpicture}
	\caption{Laufzeiten für Benchmark, der beliebige Anfangszeichen zulässt}
	\label{fig:regex_ANYdblpANY}
\end{figure}

In Abbildung \ref{fig:regex_ANYdblpANY} ist zu erkennen, dass die Ausführungszeiten aller Algorithmen bis zu einer Selektivität von ungefähr 8\% unabhängig von der Anzahl der Matches im Ergebins sind.
Je höher die Selektivität wird, desto geringer wird die Ausführungszeit.
Beide Algorithmen, welche die \emph{Flat}-Einstellung von Ragel verwenden sind im Allgemeinen schneller als die entsprechenden Gegenstücke mit der \emph{Table}-Einstellung.
Am schnellsten ist hier die \emph{LIKE}-Operation aus DogQC.
Die Verbesserung der \emph{Flat}-Variante des Musterabgleichs durch das Lane Refill beträgt zwischen 26\% und 35\% und wird für eine höhere Anzahl von Matches größer.
Für die \emph{Table}-Variante beträgt die Verbesserung zwischen 17\% und 33\% und wird ebenfalls für höhere Anzahlen von Matches größer.

\subsection{Vergleich verschiedener Automatengrößen mit dem TPC-H-Datensatz}
\label{sec:regex_evaluation_beobachtung_5}

Zum Schluss wird untersucht, welchen Einfluss die Größe des Automaten auf die beiden Algorithmen zum parallelen Musterabgleich hat.
Dazu wurde eine Selektion über die \emph{Name}-Spalte der \emph{Products}-Relation aus dem TPC-H-Datensatz durchgeführt, wobei die Größe des Automaten variiert wurde.
Zu diesem Zweck wurde ein regulärer Ausdruck erstellt, welcher ein Suchwort annimmt und beliebige Zeichen vor und nach diesem Wort zulässt
Die Selektivität dieses Ausdrucks beträgt etwa 11\%.
Um die Automatengröße zu erhöhen, wurden weitere Suchwörter, welche in dieser Form nicht im Datensatz enthalten sind, mit dem bestehenden Ausdruck \emph{oder}-verknüpft.
Dadurch wird ein größerer Automat bei gleichbleibender Selektivität erreicht.


\begin{figure}[ht]
	\centering
	\begin{tikzpicture}
		\begin{axis}[
			ybar,		% Säulendiagramm
			xlabel={Anzahl Zustände},	% Achsenbeschriftung
			xtick=data,
			xticklabels from table={daten/regex_p_name.csv}{states},	% Beschriftung der Markierungen 
			width=\textwidth,	% Diagrammbreite
			height=300, % Diagrammhöhe
			bar width=0.4cm,	% Breite der Säulen
			ylabel={Laufzeit (ms)},	% Achsenbeschritung
			ymin=0,		% Säulen stehen auf der X-Achse
			%legend style={at={(0,0)}}
			legend pos=north west,		% Legende am oberen linken Rand positionieren
			legend style={legend cell align=left} % Text linksbündig anordnen
			]
			\addplot table [x expr=\coordindex, y=flat]{daten/regex_p_name.csv}; 
			\addlegendentry{Flat};
			\addplot table [x expr=\coordindex, y=flat-buffer]{daten/regex_p_name.csv};
			\addlegendentry{Flat mit Lane Refill}
			\addplot table [x expr=\coordindex, y=table]{daten/regex_p_name.csv};
			\addlegendentry{Table}
			\addplot table [x expr=\coordindex, y=table-buffer]{daten/regex_p_name.csv};
			\addlegendentry{Table mit Lane Refill}
		\end{axis}
	\end{tikzpicture}
	\caption{Laufzeiten für unterschiedliche Automatengrößen mit dem TPC-H Datensatz}
	\label{fig:regex_p_name}
\end{figure}

Abbildung \ref{fig:regex_p_name} zeigt, dass die Ausführungszeit aller Algorithmen für größere Automaten ansteigt.
Der Unterschied zwischen 28 und 51 Zuständen fällt dabei recht klein aus, wogegen die anderen beiden Schritte einen signifikanteren Unterschied zeigen.
Eine Ausnahme bildet der Musterabgleich mit der \emph{Table}-Einstellung, welcher auch zwischen 28 und 51 Zuständen einen signifikanten Anstieg zeigt.
Die \emph{Flat}-Variante des Algorithmus ist in jedem Falle schneller als die \emph{Table}-Variante, wobei der Unterschied für größere Automaten größer wird.
Die Verbesserung durch das Einführen des Lane Refill-Verfahrens für den \emph{Flat}-Algorithmus beträgt zwischen 4\% und 6\%.
Der \emph{Table}-Algorithmus profitiert vom Lane Refill mit einer Verbesserung zwischen 0\% und 5\% mit einem Ausreißer von 20\% bei 28 Zuständen.

\section{Diskussion der Ergebnisse}

Die durchgeführten Untersuchungen ergeben ein einheitliches Bild zu dem Verhalten der vorgestellten Algorithmen und deren Verbesserungen durch das Lane Refill-Verfahren.
In den ersten drei Abschnitten wird ein Präfixtest untersucht, welcher sich für den regulären Musterabgleich und den \emph{LIKE}-Operator ähnlich wie für die zuvor beschriebene einfache Umsetzung des Präfixtests verhält.
Die geringe Laufzeit für eine niedrige Anzahl von Matches ist darin begründet, dass viele Zeichenketten vorzeitig verworfen werden können.
Für das Durchlaufen eines Automaten bedeutet dies, dass dieser in einen Fehlerzustand gerät und die Lane inaktiv wird und der \emph{LIKE}-Operator verhält sich in diesem Fall wie der einfache Präfixtest.
Dadurch, dass viele Zeichenketten vorzeitig verworfen werden, ist bei den Basisalgorithmen die Wahrscheinlichkeit geringer, dass in einem Warp ein Match vorhanden ist und somit der Warp oftmals nicht vollständig abgearbeitet werden muss.
Innerhalb der durch das Lane Refill intern gepufferten Operationen ist für geringe Selektivitäten die Anzahl der Matches pro Warp niedriger, weshalb öfter dynamisch Tupel nachgeladen werden können als für hohe Selektivitäten.
Für eine besonders geringe Anzahl von Matches erreicht das Lane Refill allerdings nicht die optimale Verbesserung, wie in Abbildung \ref{fig:regex_dblpANY_buffer} zu erkennen ist.
Dies liegt daran, dass in diesem Bereich nur vereinzelte Warps überhaupt Tupel verarbeiten müssen, die einen Match darstellen, weshalb wiederum keine große Unterauslastung auftritt, die durch das Lane Refill beseitigt werden könnte.

Für die Basisalgorithmen kann beobachtet werden, dass die mächtigen Verfahren zum regulären Musterabgleich eine weitaus geringere Performanz bieten als die funktionsärmeren Präfix- oder \emph{LIKE}-Operationen.
Dies kann damit begründet werden, dass das Durchlaufen eines Automaten mehr Instruktionen und Speicherzugriffe erfordert, als das einfache Vergleichen zweier Zeichen wie es bei den simpleren Operationen durchgeführt wird.
Die Untersuchung aus Abschnitt \ref{sec:regex_evaluation_beobachtung_2} zeigt, dass das Erhöhen der Auslastung der Warps durch das Lane Refill einen signifikanten Laufzeitgewinn erbringt, wodurch vor allem die Algorithmen, die mit regulären Ausdrücken arbeiten, profitieren und näher an die geringere Laufzeit der simpleren Operationen heran rücken.
Für einen großen Anteil von Matches wird die Verbesserung durch das Lane Refill geringer, was daran liegt, dass in den naiven Verfahren gar keine starke Unterauslastung mehr auftritt.
Die Warps sind grundsätzlich höher ausgelastet, da ein höherer Anteil Matches besteht und somit seltener der Fall eintritt, dass nur wenige Lanes arbeiten während die restlichen Lanes auf deren Fertigstellung warten.
Es lässt sich also durch das Lane Refill kein so großer Vorteil mehr erreichen.

Werden wie in Abschnitt \ref{sec:regex_evaluation_beobachtung_4} untersucht beliebige Zeichen vor dem Suchstring im regulären Ausdruck erlaubt, kann eine weitgehend konstante Laufzeit für alle Algorithmen beobachtet werden.
Dies liegt daran, dass immer die gesamte Zeichenkette durchlaufen werden muss, da der Suchstring an einer beliebigen Stelle vorkommen kann.
Die Tatsache, dass die Laufzeit in diesem Experiment für eine höhere Anzahl von Matches geringer wird, ist darin begründet, dass die in den Datensatz eingesetzten Strings, welche ein Match ergeben, eine unterdurchschnittliche Länge besitzen und somit insgesamt weniger Zeichen verarbeitet werden müssen.
Durch die Einführung des Lane Refill wird hier ein gewisser Laufzeitgewinn erreicht, da die String-Längen im Datensatz unterschiedlich sind.
Daher werden einige Lanes, die ihren String bereits abgearbeitet haben, früher inaktiv als andere und können dynamisch neue Tupel nachladen.
Bei der Verwendung des TPC-H-Datensatzes in Abschnitt \ref{sec:regex_evaluation_beobachtung_5} zeigt sich nur eine geringfügige Laufzeitverbesserung durch das Lane Refill, was daran liegt, dass die Strings alle ähnliche Längen haben und aufgrund der beliebigen Anfangszeichen vollständig durchlaufen werden müssen.

Die Laufzeit der \emph{Flat}-Einstellung von Ragel ist sowohl mit als auch ohne Lane Refill schneller als die \emph{Table}-Einstellung.
Dies ist ein Indikator dafür, dass die Größe des verwendeten Alphabets nicht zu groß für die \emph{Flat}-Variante ist und die Verwendung einer binären Suche wie sie in der \emph{Table}-Einstellung zu finden ist, nicht nötig ist, sondern zu Laufzeiteinbußen durch den erhöhten Overhead führt.
Auch die Größe des Automaten ändert an dieser Beobachtung nichts, da auch im Abschnitt 11.2.5 zu erkennen ist, dass die einfache Flat-Variante in jedem Falle schneller ist.