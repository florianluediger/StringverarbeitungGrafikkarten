\chapter{Verbesserung des einfachen String-Vergleichs}
\label{sec:equals_lane_refill}

Zur Verbesserung der Auslastung der GPU bei der Berechnung des einfachen String-Vergleichs, wird ein Verfahren vorgestellt, das auch im Kontext der kompilierten Anfragepipelines eine optimierte Laufzeit verspricht.
Dieses stellt durch Verwendung eines Puffers sicher, dass zu jeder Zeit während der Laufzeit die Auslastung eines Warps über einem bestimmten Grenzwert liegt.
Das Prinzip wurde im Kontext von kompilierten Anfragepipelines mit SIMD als \emph{consume everything} vorgestellt \cite{Lang2018} und im Kontext dieser Arbeit auf Grafikkarten angewendet.
Im Folgenden wird das Verfahren als \emph{Lane Refill} bezeichnet, da es die inaktiven Lanes in einem Warp dynamisch wieder auffüllt.

\section{Funktionsweise des String-Vergleichs mit Lane Refill}

Zur Verbesserung des einfachen String-Vergleichs, kann das naive Verfahren durch das Lane Refill erweitert werden, indem ein Puffer eingeführt wird, in dem teilweise abgearbeitete Tupel zwischengespeichert werden können.
Bei der ursprünglichen Technik gibt es drei Stellen, an denen Lanes innerhalb eines Warps inaktiv werden können.
Zum einen passiert dies, wenn beim Längenvergleich zu Beginn eine unpassende Länge festgestellt wird und zum anderen wenn beim Vergleich von zwei Zeichen ein Unterschied festgestellt wird.
Schließlich werden Lanes ebenfalls inaktiv, wenn diese ihren Vergleich abgeschlossen haben und einen passenden String gefunden haben.
Jeweils nach diesen Ereignissen soll nun überprüft werden, ob die Auslastung des Warps noch ausreichend ist, um den Grenzwert zu erreichen.
Ist dieser Grenzwert unterschritten, können wiederum zwei Fälle auftreten.
Sollten sich im Puffer noch ausreichend Tupel befinden, sodass durch Auffüllen der Grenzwert wieder erreicht ist, so werden die inaktiv gewordenen Lanes mit den Zeichenketten aus den gepufferten Tupeln befüllt und es wird weiter gerechnet.
Wurden zuvor zu wenige Elemente gepuffert, sodass der Grenzwert nicht erreicht werden kann, werden die Tupel aus den aktiven Lanes im Puffer gespeichert und die Pipeline mit frischen Tupeln neu gestartet, damit später neue Zeichenketten für den String-Vergleich bereitstehen.

\begin{figure}[ht]
	\centering
	\begin{tikzpicture}
	\node (x11) [draw,thick,minimum width=3,minimum height=3] {};
	\node (x12) [draw,thick,minimum width=3,minimum height=3, right = 0.5cm of x11] {};
	\node (x13) [draw,thick,minimum width=3,minimum height=3, right = 0.5cm of x12] {};
	\node (x14) [draw,thick,minimum width=3,minimum height=3, right = 0.5cm of x13] {};
	\node (x21) [draw,thick,minimum width=3,minimum height=3, below = 0.5cm of x11] {};
	\node (x22) [densely dashed, draw,thick,minimum width=3,minimum height=3, right = 0.5cm of x21] {};
	\node (x23) [draw,thick,minimum width=3,minimum height=3, right = 0.5cm of x22] {};
	\node (x24) [draw,thick,minimum width=3,minimum height=3, right = 0.5cm of x23] {};
	\node (x31) [densely dashed, draw,thick,minimum width=3,minimum height=3, below = 0.5cm of x21] {};
	\node (x32) [densely dashed, draw,thick,minimum width=3,minimum height=3, right = 0.5cm of x31] {};
	\node (x33) [draw,thick,minimum width=3,minimum height=3, right = 0.5cm of x32] {};
	\node (x34) [draw,thick,minimum width=3,minimum height=3, right = 0.5cm of x33] {};
	
	\node (x41) [draw,thick,minimum width=1,minimum height=1, below = 0.8cm of x31] {};
	\node (x42) [draw,thick,minimum width=3,minimum height=3, right = 0.5cm of x41] {};
	\node (x43) [draw,thick,minimum width=3,minimum height=3, right = 0.5cm of x42] {};
	\node (x44) [draw,thick,minimum width=3,minimum height=3, right = 0.5cm of x43] {};
	
	%buffer tuples
	\node (xb1) [draw,thick,minimum width=3,minimum height=3, right = 0.8cm of x44] {};
	\node (xb2) [draw,thick,minimum width=3,minimum height=3, below = 0.5cm of xb1] {};
	\node (xb3) [rotate=-90, below right = 0cm and 0.05cm of xb2] {\footnotesize $\cdots$};
	
	%buffer
	\node (rect) [draw, thick, minimum width = 0.7cm, minimum height = 2cm, below right = -0.7cm of xb1] {};
	\node[align=center, font=\footnotesize, rotate=-90, above right = 0.25cm and 0cm of rect] (b) {\textsf{divergence buffer}};
	
	%flush
	\coordinate [below = 0.2cm of x33] (fl1);
	\coordinate [below = 0.2cm of x34] (fl2);
	\coordinate [above = 0.38cm of rect] (fl3);
	
	\draw[->, thick] (x33) -- (fl1) -- (fl3) -- (rect);
	\draw[thick] (x34) -- (fl2);
	\node[align=center, font=\footnotesize, above left = 0cm of fl3] (b) {\textsf{flush}};
	
	%new elements
	\coordinate [above = 0.2cm of x41] (ne1);
	\coordinate [above = 0.2cm of x42] (ne2);
	\coordinate [above = 0.2cm of x43] (ne3);
	\coordinate [above = 0.2cm of x44] (ne4);
	\coordinate [left = 0.4cm of ne1] (ne5);
	
	
	\node (new) [above right = 0cm and -0.5cm of ne1] {\footnotesize \textsf{new elements}};
	\draw[->, thick] (ne1) -- (x41);
	\draw[->, thick] (ne2) -- (x42);
	\draw[->, thick] (ne3) -- (x43);
	\draw[->, thick] (ne5) -- (ne4) -- (x44);
	
	\node (x51) [densely dashed, draw,thick,minimum width=3,minimum height=3, below = 0.5cm of x41] {};
	\node (x52) [draw,thick,minimum width=3,minimum height=3, right = 0.5cm of x51] {};
	\node (x53) [draw,thick,minimum width=3,minimum height=3, right = 0.5cm of x52] {};
	\node (x54) [draw,thick,minimum width=3,minimum height=3, right = 0.5cm of x53] {};
	
	\node (x61) [densely dashed, draw,thick,minimum width=3,minimum height=3, below = 0.5cm of x51] {};
	\node (x62) [draw,thick,minimum width=3,minimum height=3, right = 0.5cm of x61] {};
	\node (x63) [draw,thick,minimum width=3,minimum height=3, right = 0.5cm of x62] {};
	\node (x64) [densely dashed, draw,thick,minimum width=3,minimum height=3, right = 0.5cm of x63] {};
	
	%refill 
	\coordinate [below = 0.2cm of x61] (rf1);
	\coordinate [below = 0.2cm of x64] (rf2);
	\coordinate [below = 0.23cm of rect] (rf3);
	\draw[->, thick] (rect) -- (rf3) -- (rf1) -- (x61);
	\draw[->, thick] (rf2) -- (x64);
	\node[align=center, font=\footnotesize, below left = 0cm of rf3] (b) {\textsf{refill}};
	
	\node (x71) [draw,thick,minimum width=3,minimum height=3, below = 0.5cm of x61] {};
	\node (x72) [draw,thick,minimum width=3,minimum height=3, right = 0.5cm of x71] {};
	\node (x73) [draw,thick,minimum width=3,minimum height=3, right = 0.5cm of x72] {};
	\node (x74) [draw,thick,minimum width=3,minimum height=3, right = 0.5cm of x73] {};
	
	\node[rotate=-90, below right = 0.2cm and 0.5cm of x72] (xend) {\LARGE $\cdots$};
	\node (ts) [left = 0.5cm of x11] {};
	\node (te) [left = 0.5cm of x71] {};
	\draw[->, thick] (ts) -- (te);
	\end{tikzpicture}
	\caption{Funktionsweise des Lane Refill (Quelle: Henning Funke)}
	\label{fig:divergence_buffer}
\end{figure}

In Abbildung \ref{fig:divergence_buffer} ist die Funktionsweise der vorgestellten Technik dargestellt.
Es ist zu erkennen, dass aktive Tupel in einem schlecht ausgelasteten Warp im Puffer zwischengespeichert werden, was hier als \emph{flush} bezeichnet wird.
Im Anschluss fließen neue Tupel über den Weg der Pipeline wieder in den Warp ein, wodurch dieser wieder vollständig ausgelastet ist.
Sobald erneut eine Unterauslastung auftritt, werden die zwischengespeicherten Elemente aus dem Puffer in die inaktiven Lanes geladen und der Warp ist wieder komplett ausgelastet.

Die hier vorgestellte Technik ermöglicht es, die Unterauslastung innerhalb der String-Vergleichs-Operation zu verringern, allerdings entsteht dadurch gegebenenfalls eine starke Unterauslastung bei den Folgeoperationen in der Pipeline.
Dieser Effekt entsteht, da die Überprüfung der Zeichenketten zu sehr unterschiedlichen Zeitpunkten abgeschlossen sein kann und somit immer mal wieder einige Tupel bereit, sind die Folgeoperationen der Pipeline auszuführen, während viele andere Tupel sich noch innerhalb des String-Vergleichs befinden.
Um dieses Problem zu lösen, kann beispielsweise nach dem String-Vergleich eine weitere Puffer-Operation eingeführt werden, welche sicherstellt, dass genug Zeichenketten fertig überprüft wurden, sodass die nachfolgenden Operationen mit einer ausreichenden Auslastung ausgeführt werden.

\section{Struktur des optimierten String-Vergleichs im Kernel}

Um den einfachen String-Vergleich durch Lane Refill zu optimieren, muss die Struktur des Operators im Kernel etwas angepasst werden, was in Listing \ref{equals_lane_refill_code} dargestellt wird.
Die äußere Schleife ist bereits aus der Grundstruktur einer Pipeline, wie sie in Kapitel \ref{sec:pipelining} vorgestellt wurde, bekannt und wurde hier für ein leichteres Verständnis aufgegriffen.
Innerhalb der Schleife werden wie bei der naiven Implementierung zunächst die Operatoren ausgeführt, die in der Pipeline vor dem String-Vergleich stehen und es wird die Länge der Zeichenkette untersucht und diese gegebenenfalls verworfen.

Um im nächsten Schritt beurteilen zu können, ob eine ausreichende Auslastung besteht, wird zunächst mithilfe einer Synchronisierungsoperation überprüft, wie viele Lanes im Warp aktiv sind.
Sind im Puffer und im Warp im Summe noch ausreichend aktive Elemente vorhanden, um den vorher definierten Grenzwert zu überschreiten, so kann weiter mit dem String-Vergleich fortgefahren werden.
Gegebenenfalls werden zuvor allerdings noch die leeren Lanes mit Tupeln aus dem Puffer wieder aufgefüllt, sollten die aktiven Lanes alleine nicht ausreichen, um den Grenzwert zu erreichen.
Der eigentliche Vergleich zweier Zeichen funktioniert hier wieder genau wie bei der naiven Implementierung.
Nach dem Vergleich wird geprüft, ob der String vollständig durchlaufen wurde und entsprechend die folgenden Operationen in der Pipeline ausgeführt.
An dieser Stelle fällt auf, dass im Gegensatz zu der naiven Implementierung eine Verschachtelung entsteht, da die Folgeoperationen innerhalb der \texttt{while}-Schleife des String-Vergleichs-Operators ausgeführt werden.

Ist nach erneuter Zählung der aktiven Lanes die Schleifenbedinugng nicht mehr erfüllt, da zu wenige Tupel existieren, um eine ausreichende Auslastung zu gewährleisten, werden die Tupel aus den restlichen, aktiven Lanes in den Puffer geschrieben.
Schließlich wird die Pipeline mit frischen Tupeln wieder von vorne gestartet.

\newpage

\begin{lstlisting}[language=MyC++,
caption=Struktur der String-Selektion mit Lane Refill,
label=equals_lane_refill_code]
while(!flush_pipeline) {

	/* execute previous operators in the pipeline */
	
	// if string lengths are unequal, discard
	if (active && data_length != search_length)
		active = false;
	
	int numactive = __popc(__ballot_sync(ALL_LANES, active));
	int bufferelements = 0;
	
	while(bufferelements + numactive > THRESHOLD) {
	
		if (numactive < THRESHOLD) {
		
			/* refill empty lanes from buffer in case of underutilization */
			
			bufferelements = bufferelements - numrefill;
		}
		
		// when strings don't match, inactivate the lane
		if (active && data_content[data_id] != search_string[search_id])
			active = false;
		
		search_id++;
		
		if (search_id == search_length) {
		
			/* execute following operators in the pipeline */
		
		}
		
		numactive = __popc(__ballot_sync(ALL_LANES, active));
	}
		
	if (numactive > 0) {
	
		/* flush active lanes to buffer */
		
		bufferelements += numactive;
		active = false;
	}
	
	loop_var += step;
}
\end{lstlisting}

\section{Technische Umsetzung der Pufferung}

Der Puffer wird mithilfe der CUDA-Programmierschnittstelle als Shared Memory umgesetzt, um eine effiziente Kommunikation zwischen den Lanes zu ermöglichen.
Für den einfachen String-Vergleich werden dazu zwei Speicherbereiche benötigt, welche mit dem Schlüsselwort \texttt{\_\_shared\_\_} initialisiert werden.
In einem Feld wird die Position des untersuchten Strings gespeichert und in dem anderen Feld wird der Index des Zeichens innerhalb des Strings gespeichert, das als nächstes verglichen werden muss.
Da der Shared Memory immer auf dem Level eines ganzen Blocks gültig ist, muss dieser so viele Elemente fassen können wie es Threads pro Block gibt.
Eine ausführliche Version der Umsetzung befindet sich in Anhang \ref{apx:equals_lane_refill}, hier sollen aber dennoch die Techniken zum Zwischenspeichern und Laden von Elementen kurz beschrieben werden.

\begin{lstlisting}[language=MyC++,
caption=Befüllen inaktiver Lanes mit Elementen aus dem Puffer,
label=refill_code]
if (numactive < THRESHOLD) {
	numRefill = min(32 - numactive, bufferelements);
	numRemaining = bufferelements - numRefill;
	
	previous_inactive = __popc(~__ballot_sync(ALL_LANES, active) & prefixlanes);
	
	if (!active && previous_inactive < bufferelements) {
		buf_ix = numRemaining + previous_inactive + bufferbase;
		search_id = search_id_divergence_buffer[buf_ix];
		current = current_divergence_buffer[buf_ix];
		active = true;
	}
	
	bufferelements -= numRefill;
}
\end{lstlisting}

Bei dem in Listing \ref{refill_code} dargestellten Befüllen von leer gelaufenen Lanes mit Elementen aus dem Puffer, bleiben die noch aktiven Lanes unberührt.
Die inaktiven Lanes müssen zunächst beurteilen, ob diese berechtigt sind, sich ein Tupel aus dem Puffer zuzuweisen.
Befinden sich beispielsweise zwei Elemente im Puffer, es gibt aber drei inaktive Lanes, dann erhalten die zwei Lanes mit dem niedrigeren Index ein neues Tupel und die Lane mit höherem Index bleibt weiter inaktiv.
Als erstes muss also bestimmt werden, wie viele Lanes vor der betrachteten Lane inaktiv sind.
Der Wert \texttt{prefixlanes} ist dabei eine Bitmaske mit einem Bit für jede Lane im Warp, bei der für alle vor der aktuellen Lane liegenden Lanes ein Bit gesetzt wurde.
Gibt es weniger inaktive Lanes vor der betrachteten als Elemente im Puffer, darf sich diese Lane ein neues Tupel aus dem Puffer holen.

\begin{figure}[ht]
	\centering
	\begin{tikzpicture}
	% buffer
	\node (b1) [draw, thick, minimum width=6cm, minimum height=1.5cm] {};
	\node (b2) [draw, thick, minimum width=0.75cm, minimum height=1.5cm, right = 0cm of b1] {};
	\node (b3) [draw, thick, minimum width=0.75cm, minimum height=1.5cm, right = 0cm of b2] {};
	\node (b4) [draw, thick, minimum width=0.75cm, minimum height=1.5cm, right = 0cm of b3] {};
	\node (b5) [draw, thick, minimum width=0.75cm, minimum height=1.5cm, right = 0cm of b4] {};
	\node (b6) [draw, thick, minimum width=0.75cm, minimum height=1.5cm, right = 0cm of b5] {};
	\node (b7) [draw, thick, minimum width=1.75cm, minimum height=1.5cm, right = 0cm of b6] {};
	
	% warp
	\node (w1) [draw, thick, minimum width=0.75cm, minimum height=0.75cm, below right = 1.5cm and -0.1cm of b1.south] {};
	\node (w2) [draw, thick, minimum width=0.75cm, minimum height=0.75cm, right = 0cm of w1, fill=black!30] {};
	\node (w3) [draw, thick, minimum width=0.75cm, minimum height=0.75cm, right = 0cm of w2, fill=black!30] {};
	\node (w4) [draw, thick, minimum width=0.75cm, minimum height=0.75cm, right = 0cm of w3] {};
	\node (w5) [draw, thick, minimum width=0.75cm, minimum height=0.75cm, right = 0cm of w4] {};
	\node (w6) [draw, thick, minimum width=0.75cm, minimum height=0.75cm, right = 0cm of w5, fill=black!30] {};
	\node (w7) [draw, thick, minimum width=0.75cm, minimum height=0.75cm, right = 0cm of w6] {};
	\node (w8) [draw, thick, minimum width=0.75cm, minimum height=0.75cm, right = 0cm of w7] {};
	
	% arrows to warp
	\coordinate [below = 0.3cm of b4.south] (arr11);
	\coordinate [above = 0.6cm of w2.north] (arr12);
	\draw [->, thick, dashed, rounded corners=0.4cm] (b4.south) -- (arr11) -- (arr12) -- (w2.north);
	
	\coordinate [below = 0.4cm of b5.south] (arr21);
	\coordinate [above = 0.5cm of w3.north] (arr22);
	\draw [->, thick, dashed, rounded corners=0.4cm] (b5.south) -- (arr21) -- (arr22) -- (w3.north);
	
	\coordinate [below = 0.5cm of b6.south] (arr31);
	\coordinate [above = 0.4cm of w6.north] (arr32);
	\draw [->, thick, rounded corners=0.3cm] (b6.south) -- (arr31) -- (arr32) -- (w6.north);
	
	% offset ticks
	\coordinate [above = 0.5cm of b1.north west] (tick1);
	\coordinate [above = 0.5cm of b2.north west] (tick2);
	\coordinate [above = 0.5cm of b4.north west] (tick3);
	\coordinate [above = 0.5cm of b6.north west] (tick4);
	
	\draw [thick] (b1.north west) -- (tick1);
	\draw [thick] (b2.north west) -- (tick2);
	\draw [thick] (b4.north west) -- (tick3);
	\draw [thick] (b6.north west) -- (tick4);
	
	% offset arrows
	\coordinate [below = 0.25cm of tick1] (arrow1);
	\coordinate [below = 0.25cm of tick2] (arrow2);
	\coordinate [below = 0.25cm of tick3] (arrow3);
	\coordinate [below = 0.25cm of tick4] (arrow4);
	
	\draw [->, thick] (arrow1) -- (arrow2);
	\draw [->, thick] (arrow2) -- (arrow3);
	\draw [->, thick] (arrow3) -- (arrow4);
	
	% labels
	\node [above = 0.3cm of b1.north] {\textsf{bufferbase}};
	\node [above = 0.3cm of b3.north west, align=center] {\textsf{num}\\\textsf{remaining}};
	\node [above = 0.3cm of b5.north west, align=center] {\textsf{previous}\\\textsf{inactive}};
	
	\node [right = 0.3cm of b7] {\textsf{buffer}};
	\node [right = 0.3cm of w8] {\textsf{warp}};
	
	\end{tikzpicture}
	\caption{Berechnung des Indexes für ein Element im Puffer}
	\label{fig:buffer_index}
\end{figure}

Um das richtige Element zu erhalten, muss zunächst wie in Abbildung \ref{fig:buffer_index} dargestellt dessen Index im Puffer errechnet werden, welcher sich aus verschiedenen Offsets Zusammensetzt.
\texttt{buffer\_base} bestimmt dabei die Position des Speicherbereichs, welcher dem aktuellen Warp im Block zugewiesen wurde.
Da der Puffer grundsätzlich von rechts abgearbeitet wird, muss noch der Wert \texttt{num\_remaining} für die Elemente, die im Puffer verbleiben sollen, addiert werden und schließlich noch die Tupel übersprungen werden, die für Lanes mit niedrigerem Index bestimmt sind.

Ist der Index berechnet, können der String und der Suchindex innerhalb des Strings aus dem Puffer geladen werden und die Lane wieder aktiv geschaltet werden.

\begin{lstlisting}[language=MyC++,
caption={Auslagern übriger, aktiver Lanes in den Puffer},
label=flush_code]
if (numactive > 0) {
	previous_active = __popc(__ballot_sync(ALL_LANES, active) & prefixlanes);
	buf_ix = bufferbase + bufferelements + previous_active;
	
	if(active) {
		search_id_divergence_buffer[buf_ix] = character_index;
		current_divergence_buffer[buf_ix] = current;
	}
	
	bufferelements += numactive;
	active = false;
}
\end{lstlisting}

Das in Listing \ref{flush_code} dargestellte Verfahren zum Einlagern übriger, aktiver Lanes in den Puffer funktioniert ähnlich wie das wieder befüllen von Lanes.
Zunächst wird wieder der Index berechnet, auf den die aktuelle Lane im Puffer zugreifen kann, was identisch zu dem in Abbildung \ref{fig:buffer_index} dargestellten Verfahren funktioniert, nur dass in diesem Falle die aktiven Lanes statt der inaktiven Lanes als Offset verwendet werden.
Alle verbleibenden, aktiven Lanes speichern daraufhin die Position ihres vorher untersuchten Strings und den als nächstes zu untersuchenden Index innerhalb des Strings im Puffer ab.
Schließlich wird noch die Anzahl der Elemente im Puffer erhöht und der Algorithmus kann anschließend wie gehabt verfahren.

\section{Reduzierung des Overheads}
\label{sec:unroll}

Ein Nachteil dieses Verfahrens liegt darin, dass durch das Verwalten des Puffers ein erhöhter Overhead entsteht, der die bessere Performanz durch eine gute Auslastung der GPU überschatten könnte.
Um diesen Overhead zu verringern, könnte es sinnvoll sein in größerem Zeitabstand die Auslastung zu überprüfen und entsprechend Lanes neu zu befüllen.
Um dies zu erreichen, muss wie in Listing \ref{unroll_code} lediglich der Zeichen-Vergleich in einer kurzen Schleife laufen, sodass dieser ohne Unterbrechung einige male nacheinander ausgeführt wird.
Dabei muss natürlich nach jeder Überprüfung auch geschaut werden, ob der String bereits fertig durchlaufen wurde, damit gegebenenfalls die Folgeoperationen in der Pipeline ausgeführt werden können.
Durch das Schlüsselwort \texttt{\#pragma unroll}, ersetzt der Präprozessor die einfache Schleife durch Duplikate des Codes, wodurch eine weitere, kleine Leistungssteigerung erreicht wird.

\begin{lstlisting}[language=MyC++,
caption={Reduzierung des Overheads des Verfahrens},
label=unroll_code]
#pragma unroll
for (int u = 0; u < ITERATION_COUNT; u++) {
	int data_id = search_id + char_offset[current];
	
	// when strings don't match, inactivate the lane
	if (active && data_content[data_id] != search_string[search_id])
		active = false;
	
	search_id++;
	
	if (search_id == search_length)
		break;
}
	
if (search_id == search_length) {

	/* execute following operators in the pipeline */
	
	active = false;
}
\end{lstlisting}
